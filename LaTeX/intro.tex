%présenter le sujet, ce qu'on va y faire (etc)
Dans le cadre de nos études, nous avons dû développer un site internet à destination du monde étudiant.
C'est dans la matière ITW \textit{(introduction aux technologies du web)} que ce mini-projet est mis en place.
\\
Le site doit être statique et non-responsive. Il doit comporter plusieurs pages, notamment :
\begin{itemize}
    \item page d'accueil,
    \item page actualité,
    \item page logement,
    \item page voyage,
    \item page réseaux sociaux,
    \item page contact,
    \item page mention légale
\end{itemize}

\vspace{1\baselineskip}

Nous avons choisi de faire notre page sur un sujet "sur-réaliste" : un monde étudiant inter-galactique en immersion dans le monde de \textit{Star Wars Story}.
C'est-à-dire qu'il existerait plusieurs planètes habitables dans différents systèmes galactiques, où chaque planète à une spécialité d'étude.\\


Les actualités comprennent des évènements présents sur les planètes, des informations géopolitiques, de nouvelles formations disponibles\ldots

Les logements proposés vous permettent d'observer l'architecture de chaque planète, leur prix, les types de séjours que vous pourrez réserver\ldots

Les offres de voyages montrent les exclusivités, les activités\ldots \\


Ce site web est bien évidemment factice et nous avons dû respecter plusieurs consignes concernant l'aspect global des pages, les technologies utilisés, le contenu \ldots